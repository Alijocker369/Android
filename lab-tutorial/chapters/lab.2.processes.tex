\chapter{Lab 2 - Processes}

Firstly, congratulations on completing the first lab! Hopefully you have had a chance to play
around with the MOS kernel and have a better understanding of its structure.

\begin{note}
    \item Do email me or ask the lab assistant for help if you are stuck.
\end{note}

In this lab, we will be looking at the process management system in MOS.

\section{Process Control Blocks}

As you have seen in the lecture, `a process is a program in execution', and a process control
block (PCB, in short) is a data structure that contains information about a process.

Now, let's look at the PCB structure in MOS.

The PCB structure is defined in \texttt{kernel/include/mos/tasks/task\_type.h}:

\begin{verbatim}
typedef struct process_t
{
    u32 magic;
    const char *name;
    pid_t pid;
    process_t *parent;
    terminal_t *terminal;
    uid_t effective_uid;
    paging_handle_t pagetable;

    ssize_t files_count;
    io_t *files[MOS_PROCESS_MAX_OPEN_FILES];

    ssize_t threads_count;
    thread_t *threads[MOS_PROCESS_MAX_THREADS];

    ssize_t mmaps_count;
    proc_vmblock_t *mmaps;
} process_t;
\end{verbatim}

The very first field is a magic number, which is used to check if we are dealing with a valid
PCB. Then follows the process name and its PID.

Each process has a parent, which is the process that created it, and this forms a tree-like
structure.

\begin{note}
    \item At the root of the tree is the \texttt{init} process, it's fun to think about
    it being its own parent.
\end{note}

You may have noticed that the PCB MOS uses is a bit different from the one in the lecture.
It doesn't specify the process state or its stack/heap/text memory regions. This is because
MOS is using threads as the basic unit of execution, and each thread has its own state,
you'll find out more about threads in the next lab session.

\begin{exercise}
    \item Read the \texttt{kernel/include/mos/tasks/task\_type.h} header file to get
    familiar with the PCB structure in MOS.
\end{exercise}

\section{Process Creation and Termination}

A process has to be created from a `program' (a binary executable file) in order to be
executed.
