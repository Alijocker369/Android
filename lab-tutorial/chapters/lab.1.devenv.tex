\chapter{Lab 1 - Setting Up the Development Environment}

MOS is an operating system, thus, preparing for a fully-functional development
environment is not an easy task (bruh). Several efforts have been made to make
the process easier.

To set up the development environment for MOS (well, for now, only x86 target exists),
several tools are required:

\begin{itemize}
    \item CMake and (Ninja or Make)
    \item i686-elf-binutils, i686-elf-gcc and i686-elf-gdb
    \item NASM
    \item cpio
    \item qemu-system-i386
\end{itemize}

We'll discuss about each of them in the following sections.

\section{CMake and (Ninja or Make)}

CMake is a cross-platform build system generator, which generates (e.g. Makefiles)
for a build system (e.g. GNU Make) to use.

MOS uses CMake instead of GNU Make directly, because CMake is more flexible and
powerful in many aspects.

We'll come back to CMake in the later chapters.

\section{i686-elf-\{binutils,gcc,gdb\}}

As its name suggests, this is a cross-compiler toolchain for `i686-elf' target.
`i686' means the 32-bit x86 architecture, and `elf' means the executable format.

\begin{tip}
    Most 64-bit Linux has a target triple of `x86\_64-pc-linux-gnu'.
\end{tip}

Unlike other applications (e.g. bash or vim) that they run on an existing operating
system and a standard libc (say, glibc or musl). MOS itself is the operating system,
thus there's not an existing OS for it to run on, neither a standard libc (i.e. no
\texttt{printf}, no \texttt{malloc} etc.) for it to use.

A `bare-metal' compiler toolchain is exactly for this situation. Considering you're
directly interacting with the CPU and the hardware.

There are majorly two ways to get this toolchain, either by building it from source
or by downloading a pre-built binary.

\subsection{Downloading a pre-built binary}

If you don't want to build the toolchain from source, you can download a pre-built
binary from \href{https://github.com/moodyhunter/i686-elf-prebuilt/releases}{moodyhunter/i686-elf-prebuilt} (choose the i686 one).

\begin{warning}
    \begin{itemize}
        \item Using pre-built binary saves time, but please consider doing so \textbf{only} if you trust the author.
        \item The above pre-built binary is built with GitHub Actions, and is built on Ubuntu 20.04.5 LTS (Image \texttt{ubuntu-20.04} version \texttt{20221027.1}).
    \end{itemize}
\end{warning}

\subsection{Building From Source}

The source code of binutils and gcc can be found at \href{https://www.gnu.org/software/binutils}{GNU Binutils's Website}
and \href{https://gcc.gnu.org}{GNU GCC's Website} respectively.

\begin{note}
    For Arch Linux users, checkout
    \href{https://github.com/moodyhunter/repo/blob/main/moody/i686-elf-binutils/PKGBUILD}{i686-elf-binutils},
    \href{https://github.com/moodyhunter/repo/blob/main/moody/i686-elf-gcc/PKGBUILD}{i686-elf-gcc} and
    \href{https://github.com/moodyhunter/repo/blob/main/moody/i686-elf-gdb/PKGBUILD}{i686-elf-gdb}.
\end{note}

The script located at \texttt{docs/assets/i686-elf-toolchain.sh} downloads, compiles and installs them into the
directory specified by the \texttt{PREFIX} variable.

\lstinputlisting[language=Octave]{assets/i686-elf-toolchain.sh}

\section{NASM}

NASM is an assembler for x86 architecture. There are several files under `arch/x86`
that are written in NASM.

NASM can be installed via your Linux's package manager, (e.g. `sudo apt install nasm`
or `sudo pacman -S nasm`). Or you can download a pre-built binary from [NASM's website](https://www.nasm.us/).

\section{cpio}

cpio is the tool to create archives in a the cpio format. MOS uses cpio as the initial
root filesystem.

Similar to NASM, you can install cpio via both package manager like `sudo apt install cpio`
or `sudo pacman -S cpio`, downloading and compiling it from source is also possible, but is not discussed here.

\section{qemu-system-i386}

QEMU is an open-source emulator, it also provides a gdb stub for debugging.

QEMU can be installed via your Linux's package manager, (e.g. `sudo apt install qemu-system-i386`
or `sudo pacman -S qemu-system-x86`). See its [download page](https://www.qemu.org/download) for more details.
