\chapter*{Introduction to the MOS Lab}

\textbf{Welcome to the MOS Lab!}

MOS is a microkernel operating system, it's designed to be simple and easy to understand.
During the lab, you will be working with the MOS kernel and exploring corresponding concepts of
operating systems.

Throughout the labs, you will explore the following topics, in the order listed. Items in
\textbf{bold} are the main topics of the lab, which you may be asked to implement something for.

\begin{warning}
    \item The table below is still a work in progress, efforts are being made to make it match the
    timetable of the lectures.
\end{warning}

\begin{itemize}
    \item \textbf{Lab 1}: Basic Compiler and Linker Usage
          \subitem Cross-Compiler Usage
    \item \textbf{Lab 2}: Process and Threads
          \subitem \textbf{PCB} \textbf{TCB}, process tables and thread tables
          \subitem \textbf{Creating} a new process
          \subitem \textbf{The famous \texttt{fork()} syscall}
    \item \textbf{Lab 3} Multi-Threading
          \subitem \textbf{Multi-Thread} applications
          \subitem \textbf{Signals} handling
    \item \textbf{Lab 4}: Scheduling
          \subitem \textbf{Scheduler} and different scheduling algorithms
    \item \textbf{Lab 5}: Synchronization
          \subitem \textbf{Synchronization Primitives} such as \texttt{mutex} and \texttt{semaphore}
          \subitem A demonstration of \textbf{Deadlock}s
    \item \textbf{Lab 6}: Memory Management
          \subitem Segmentation and \textbf{Paging}, \textbf{Virtual Memory} and \textbf{Address Spaces}
          \subitem Common memory management practices in modern kernels
\end{itemize}
